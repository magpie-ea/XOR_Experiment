\documentclass[11pt,letterpaper]{article}
\usepackage[margin=1in]{geometry}
\usepackage[utf8]{inputenc}
\usepackage[T1]{fontenc}
\usepackage{hyperref}
\renewcommand{\familydefault}{\sfdefault}
\usepackage{helvet}
\pagestyle{empty}
\usepackage[kerning=true]{microtype}
\usepackage{parskip}
\usepackage{sansmath}
\usepackage{graphicx}
\usepackage{sidecap}  
\sidecaptionvpos{figure}{c}
\usepackage{float}
\usepackage{color, soul}
% Feel free to use additional packages for glosses, figures, whatnot.
\usepackage[dvipsnames]{xcolor}
\newcommand{\mf}[1]{\textcolor{BurntOrange}{[MF: #1]}}
\newcommand{\pt}[1]{\textcolor{Cerulean}{[PT: #1]}}
\newcommand{\bvt}[1]{\textcolor{ForestGreen}{[BvT: #1]}}
% The next bit is for reserving sufficient space for authors,
% affiliations, and e-mail address.  No need to change for initial
% anonymous version.  For the final version, replace the
%\toggletrue{anonymous} %with \togglefalse{anonymous} to de-anonymize.
\usepackage{etoolbox}
\newtoggle{anonymous}
\toggletrue{anonymous}

\renewcommand{\title}[1]{\textbf{#1}\\}
\newcommand{\authors}[1]{\iftoggle{anonymous}{\phantom{#1}}{#1}\\}
\newcommand{\email}[1]{\iftoggle{anonymous}{\phantom{#1}}{#1}}

\begin{document}

% First page:

% Insert title, authors, affiliations, and e-mail address in the next three lines:

\title{The role of relevance, competence and priors for scalar implicatures}
\authors{Michael Franke (University of T\"ubingen), Bob van Tiel (Radboud University), Polina Tsvilodub (Osnabr\"uck university)}
\email{pstvilodub@uos.de}

% Here goes the main text of your abstract:

If someone says "Anna ate some cookies." the hearer of the sentence might infer that Anna ate \textit{some, but not all} cookies that were in the jar. %Similarly, when hearing the sentence "Donald ate a donut or a pretzel." one might infer that Donald ate either \textit{the donut or the pretzel, but not}. 
Such an inference is an instance of \textit{scalar implicature (SI)}---an inference based on the comparison to a salient informationally stronger alternative ``all'' (Horn, 1972). That is, the listener might arrive at the upper-bounded reading of ``some'' by reasoning about a cooperative speaker: if the stronger alternative was true, a rational agent would have used the stronger utterance; but since she did not, she must have lacked the evidence for using ``all'' (Grice, 1975). Current literature also suggests an analogous account for the exclusive interpretation of ``or'', generated by comparison to the stronger alternative ``and'' (Geurts, 2010).

Prior work addresses the influence of sentence structure, or of specific linguistic markers (Li, 2021). Futhermore, literature suggests that, among others, three factors influence the robustness of scalar implicatures: (1) \textit{relevance} of the stronger alternative to the listener, (2) the \textit{competence} of the speaker about the truth of the stronger alternative, and (3) the \textit{prior probability that the stronger alternative is true} (Sperber\&Wilson, 1995; Goodman\&Stuhlmüller, 2013; Degen, Tessler\&Goodman, 2015). However, their influence was mostly investigated for individual factors. We set out to explore how these three factors influence the robustness of scalar implicatures in combination, when the SI triggers ``some'' and ``or'' are presented in rich context. Simultanously, we directly compare these two triggers with respect to their influenceability by these factors.

We conducted a web-based rating study, whereby participants read background stories which were designed to score either high or low with respect to each factor of interest, manipulated within-subjects (high/low prior probability of the more informative alternative $\times$ high/low speaker competence $\times$ high/low contextual relevance of the alternative; eight unique conditions, applied to each trigger). 
On critical trials, participants were asked to rate four sentences on sclae ranging from ``certainly true'' to ``certainly false'' (converted to 0--100), one per factor and one containing an upper-bounded (or exclusive) paraphrase of the trigger. The sentences elicited judgements of how likely it is that each factor was high, and how likely the critical scalar implicature was true (s. transcript). If the SI account is true, on average, we expect higher likelihood ratings for the scalar implicature, if the relevance and comptenece in the item were judged as high and the prior of the alternative as low. Each participant saw eight stories (four per trigger) sampled from 32 stories per trigger, randomly shuffled with visually identical attnetion checks.

We recruited 277 participants through Prolific (71  excluded following preregistered exclusion criteria). The rating were z-scored within each factor for each participant. We analysed the results using a Bayesian linear mixed effects model, regressing the target implicature likelihood rating against the fixed effects of predictor ratings (i.e., the relevance, comptence and prior ratings elicited by the same participant for that vignette), the effect of trigger and their two-, three-, and four-way interactions. Random intercepts and random slope effects for the main effects of trigger, relevance, competence and prior by-subject, as well as random intercepts by-vignette were included.

Consistent with predictions, for the trigger ``some'', we find a clear negative prior probabilty effect of the stronger alternative ``all'' being true (probability of the effect of prior being smaller than 0.05: $P = 0.999$). Similarly, we find a clear positive effect for speaker competence (effect larger than 0.05 probability: $P =  1$). We do not find any clear effects of relevance. For the trigger ``or'', we only find a positive effect of competence ($P =  0.993$). Further, we computed exploratory pairwise correlations of all the predictors. Interestingly, we find a significant correlation between the prior and the relevance effects for the trigger ``or'' ($R^2 = -0.106$). We also found a significant correlation between competence and relevance ($R^2 = 0.127$) for ``or''. No correlations were found for ``some''. 
Overall, the results provide moderate evidence for the scalar implicature account of ``some'' and ``or'', but also indicate that there might be intricate dependencies between the factors' influence on the robustness of SI, especially for disjunction interpretation.
\newpage

% Second page for additional materials, example stimuli, graphs, tables, references:


%\begin{figure}
%   \makebox[\linewidth][c]{\includegraphics[width=0.7\linewidth]{amlap2020_E1-E2.pdf}}%
% \caption{E1 (left): Means and 95\% bootstrapped confidence intervals (in all plots) of ratings (on a scale ranging 0-100) for how well a sentence with the adjective “big” (or “small”) describes a member of a subordinate category (normal-sized compared to other members of the same category in the context picture) when different nouns (color) appear in different syntactic frames (x-axis). Points represent participant means within condition.\newline 
%    E2 (right): Basic-level referent label productions (e.g., “dog” when the referent is a normal-sized Great Dane) in different syntactic frames (x-axis).}
%    \label{syntax-production}
%\end{figure}


%\begin{figure}
   
%    \makebox[\linewidth][c]{\includegraphics[scale=0.7]{amlap_expt3-cc-inference.pdf}}%
    
%    \caption{Inferred basic-level comparison classes (e.g. “...big relative to other \textit{dogs}”) when the referent (e.g., a Great Dane) appears in basic-level visual context (left panel) or in subordinate context (right panel) from a sentence where the N is “one”, the subordinate (“Great Dane”) or the basic-level (“dog”) label of the referent (colors), appearing in different syntactic positions (x-axis).}
%    \label{cc-infer}
%\end{figure}
%\begin{SCfigure}[1.7]
%    \includegraphics[ scale=0.5]{amlap_expt4_double_mod2.pdf}
 %   \caption{Pilot Results: Inferred basic-level comparison classes (e.g. ‘...big relative to other \textit{dogs}’) when the directly modified subordinate N (‘big Great Dane’) appears in different syntactic positions (x-axis).\newline
%    \newline \textbf{References:} [1] Kamp, J., In \textit{Formal semantics of natural language}, 1975  [2] Kennedy, C., \textit{Linguistics and Philosophy, 30(1)}, 2007 [3] Tessler, M. H.; Lopez-Brau, M.; Goodman N. D., In \textit{39th annual meeting of the Cognitive Science society}, 2017
%    \label{double-mod}[4] Reboul, A., In \textit{Language typology and language universals, an international handbook, vol.1}, 2001}
%\end{SCfigure}

\newpage

% Optional third page for additional information if the investigated language is not English:
\end{document}
